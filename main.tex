% main.tex
%
% Hauptdatei der Bsc Arbeit
%
% Dominic Dauer, November 2017

\documentclass[a4paper, halfparskip]{article}
\usepackage[utf8]{inputenc}
\usepackage[ngerman]{babel}

%\usepackage[draft]{graphicx} % Abb. draft entfernen = Bilder sichtbar 
%\usepackage{capt-of} % Abbildungs- und Tabellenbeschriftungen in minipage
%\usepackage{pdfpages} % Dient dem Einfügen von PDFs
%\usepackage{amsmath} % Formeln
%\usepackage{xcolor} % Farbige Tables
%\usepackage{booktabs} % Eine Erweiterung der Typografie (Fette hlines,...)
%\usepackage{geometry}

% Jurabib für Quellenangaben, \bibliographystyle{jurabib} nicht vergessen!
\usepackage[
			ibidem=strict,
			commabeforerest,
			see,
            titleformat={all,colonsep,italic},
            authorformat={all,abbrv},
            pages=format
]{jurabib} 

\renewcommand\samepageibidemname{Ebd.}
\AddTo\bibsgerman{
	\renewcommand*{\ibidemname}{Ebd.}
	\renewcommand*{\ibidemmidname}{Ebd.}
} % Ebd. statt ibid.

% hochgestelltes Textregistered
\def\SymbReg{\textsuperscript{\textregistered}}

\parindent 0pt  % Absätze nicht einrücken ...
\parskip\medskipamount  % ... sondern durch einen kleinen Umbruch ersetzen

\begin{document}

\tableofcontents
\thispagestyle{empty}
\pagebreak

\begin{abstract}
Das Bohrwiderstandsverfahren wurde entwickelt um den Holzzustand entlang eines
Bohrkanals zu ermitteln. Das Verfahren kann sowohl an stehendem, grünen Holz als
auch an verbauten, trockenem Holz angewendetet werden. 
In dieser Arbeit wird das Bohrwiderstandsverfahren mit einem
Resistographen\SymbReg angewandt um Jahrringanalysen an stehendem Holz
durchzuführen. Dahingehend wird es auf seine Praktikabilität überprüft um 
ein neues Einsatzgebiet der Bohrwiderstandsmessung zu eröffnen.
\end{abstract}

% Abbildungsverzeichnis
\addcontentsline{toc}{section}{Abbildungungen}
\listoffigures

% Tabellenverzeichnis
\addcontentsline{toc}{section}{Tabellen}
\listoftables

\section{Einleitung und Themenstellung}\label{sec:einleitung}
% TODO Quelle für ersten Absatz finden
Das Bohrwiderstandsverfahren wurde entwickelt um den Holzzustand
anhand der an der Bohrnadel gemessenen Dichte entlang des Bohrkanals zu
erfassen. Heute wird es sowohl für stehendes Holz in der Baumuntersuchung zu
Zwecken der Verkehrssicherheit\footcite{rinn:bohrwiderstandsprofile}, als auch
für trockenes und bereits verbautes Holz im Bereich der
Holzkonstruktionsuntersuchung
angewendet\footcite{rinnfischer:zustandskartierung}. In dieser Arbeit wird das
Bohrwiderstandsverfahren in dem Bereich der Jahrringanalyse angewandt und
dahingehend auf seine Praktikabilität überprüft.

Jahrringanalysen an stehendem Holz werden in der Waldwachstumskunde unter
anderem zur Untersuchung von klimatischen Einflüssen auf einen Bestand
herangeszogen. Anhand von erstellten Jahrringkurven, die mit einer
Jahrringchronologie verglichen werden, können unter Zuhilfenahme von
klimatischen Daten, Aussagen über das Wachstum eines Bestandes unter
bestimmten Einflüssen, beispielsweise einer ausgeprägten Trockenperiode,
angestellt werden. Eine Jahrringkurve wird über das zählen von Jahrringen und
das messen der Jahrringbreiten erstellt. Dies geschieht auf Grundlage von
Holzproben, die typischerweise als Holzquerschnitt oder als Bohrkern
vorliegen. Liegt die Probe als Querschnitt vor, musste zuvor ein Baum gefällt
werden. Liegt die Probe als Bohrkern vor, wurde ein stehender Baum durch
das Ziehen eines Bohrkerns mittels Zuwachsbohrers (Verdrängungsbohrer),
verletzt. Beide Verfahern haben die negative Auswirkung, dass der Baum
entweder stirbt bzw. schon tot ist oder, dass ein lebender Baum aus
forstlicher Sicht entwertet wird und eine nicht unerhebliche Verletzung
im Holz entsteht, die eine Eintrittspforte für holzabbauende Pilze
bildet\footcite[66]{schweingruber1983jahrring}.

% TODO Abbildung zu Jahrringen zusammen mit Bohrung einfügen
In dieser Arbeit wird das Verfahren der Bohrwiderstandsmessung im Bereich der
Jahrringanalyse angewendet. Es wird überprüft ob Messungen mittels
Bohrwiderstandsmessgerätes an stehendem Holz ausreichen um Jahrringkurven zu
erstellen und somit Aussagen zu oben erwähnten Themen, wie die Auswirkung von
klimatischen Ereignissen auf das Wachstum eines Bestandes, zu machen. Die
Messung eines Bohrwiderstandsmessgerätes liefert bei ausreichender
Genauigkeit, korrekter Anwendung und einer ausreichend hohen Korrelation
($r>0.9$) der Bohrkurve zur Holzdichte entlang des Bohrkanalas, eine
Bohrkurve, die Aussagen zum Holzzustand
zulassen\footcite[1]{rinn:resi_drill_transition}. Auf dieser Kurve sind i.d.R.
bereits Jahrringe, sowohl Jahrringgrenzen zusammen mit Früh und Spätholzzonen,
als auch intraannuelle Jahrringstrukturen zu erkennen (siehe Abb. XXX). Ob
diese Messergebnisse ausreichen um Jahringkurven zu erstellen, die in ihrer
Zuverlässigkeit, Genauigkeit und im Aufwand zur Erstellung, mit den
Jahrringkurven basierend auf den o.g. etablierten Verfahren mithalten können
ist Gegenstand dieser Arbeit.

% TODO Quelle für den letzten Satz
Sind Messungen zu o.g. Zwecken und unter den genannten Gesichtspnkten mit
einem Bohrwiderstandsmessgerät möglich, könnten Bestandesbäume und
Holzkonstruktionen mit minimalem Schadensaufwand dendrochronologisch
untersucht werden, da eine Bohrwiderstandsmessung ein Loch im Holzkörper von
nicht mehr als 3mm Durchmesser erzeugt und den Baum damit nicht wesentlich
schädigt.

\section{Material und Methoden}\label{sec:material}
\subsection{Bohrwiderstandsverfahren}\label{subsec:bohrwiderstandsverfahren}
In dieser Arbeit wurde das Bohrwiderstandsgerät Resistograph\SymbReg\ Modell
R6 der Firma Rinntech für alle Bohrwiderstandsmessungen verwendet. Im
Folgenden wird ein kurzer Überblick über die Entstehungsgeschichte, den Aufbau
und die Bedienung des Gerätes gegeben.  

\subsubsection{Geschichte der
Bohrwiderstandsmessung}\label{subsubsec:geschichte_bwm}
Der von Frank Rinn entwickelte Resistograph\SymbReg\ ist ein elektronisch
gesteuertes Bohrwiderstandsmessgerät, dessen zugrunde liegendes Verfahren 1990
von Frank Rinn im Patent "Vorrichtung zur Materialprüfung, insbesondere
Holzprüfung durch Bohr- bzw. Eindringwiderstandsmessung"{} patentiert
(Patentnr.: DE 4122494 B4) wurde.

Das Bohrwiderstandsverfahren hat eine lange Geschichte, welche in den 1970ern
mit den deutschen Ingenieuren Kamm und Voss begann. Die beiden Ingenieure
entwickelten ein Gerät mit einer durch einen Elektromotor angetriebenen
Bohrnadel, mit welchem es möglich war, in Holz einzudringen. Ein
federgelagerter Aufzeichnungsapperat zeichnete ein eins zu eins Profil über
einen Kratzstift auf Wachspapier. Der federgelagerte Aufzeichnungsmechanismus
erzeugte aufgrund der Resonanzschwingungen der Feder nicht verwertbare
Ergebnisse. 

Die Weiterentwicklung des Geräts war ein elektronisch messendes
Bohrwiderstandsgerät, welches den gemessenen Bohrwiderstand mittels eines
akustischen Signals anhand des Energieverbrauchs der beim Bohren entsteht
wiedergab.

Frank Rinn entwickelte 1987 im Rahmen seiner Diplomarbeit das von der Firma
FEIN in Stuttgart produzierte und elektrisch aufzeichnende
Bohrwiderstandsmessgerät "Densitomat", welches auch an Wissenschaftler und
Experten weltweit verkauft wurde. Aufgrund der hohen Korrelation (r
\textgreater{} 0.9) der Bohrprofile mit der Dichte des Holzes entlang des
Bohrkanals, waren das erste mal Aussagen über den Zustand des Holzes möglich.

Der Name "Densitomat" war kein eingetragener Markenbegriff. Und so kam es zur
Weiterentwicklung des "Densitomaten" zum Resistograph\SymbReg, dessen Name
eine eingetragene Marke darstellt. Resistograph\SymbReg-Geräte charakterisiert
eine besonders hohe Genauigkeit in der Auflösung und Korrelation der Messung
zur Dichte, sodass sehr exakte Aussagen über den Holzzustand möglich
sind.\footcite{rinn:resi_drill_transition}

Im Folgenden wird der Aufbau des Resistograph\SymbReg\ und dessen praktische
Benutzung am Holzkörper beschrieben. Zusätzlich wird die Auswertung der
Messkurven am Computer erläutert. 

\subsubsection{Aufbau des Resistograph\SymbReg}
Der Resistograph der Firma Rinntech ist ein elektronisch gesteuertes
Bohrwiderstandsmessgerät zur Untersuchung von grünem und trockenem
Holz\footcite{rinn:risserkennung}. Im Folgenden wird der Resistograph\SymbReg\
Modell R6 vorgestellt, mit dem der Verfasser gearbeitet hat.

Das Gerät besitzt zwei Motoren. Der eine Motor ist für den Vortrieb der Nadel
ins Holz zuständig und sitzt im Handgriff. Der andere Motor sorgt für die
Rotierung der Nadel und ist im Schlitten untergebracht, der durch die Kraft
des ersten Motors nach vorne oder zurück bewegt werden kann. Es ist möglich,
bis zu 50 cm tiefe Bohrungen anzufertigen. Mittels einer sehr fein abgestuften
Sensorik wird beim Hineinbohren in einen Holzkörper die Dichte erfasst. Die
Software reguliert in Abhängigkeit der Härte des Holzes die
Vorschubsgeschwindigkeit. Über einen via Bluetooth verbundenen Drucker lässt
sich noch vor Ort eine eins zu eins mit der Bohrtiefe korrelierende
Dichtekurve ausdrucken. Die Kurve wird zusätzlich auf dem geräteinternen
Speicher zum späteren Auslesen via USB und Bearbeiten am Computer gespeichert.
Der Speicher ermöglicht das Speichern von bis zu 10.000 Bohrkurven. Darüber
hinaus ist es auch möglich, das Bohrprofil mit einer App für Smartphones
aufzuzeichnen und abzubilden.

Auf der Unterseite des Gerätes sind Bedienknöpfe angebracht, die auch mit
dicken Handschuhen bedient werden können. Mittels der Bedienknöpfe sind alle
notwendigen Optionen, wie beispielsweise die Einstellung der
Projektnummerierung, die Bohrtiefe und -geschwindigkeit einstellbar.

Die obere Hälfte des R6 besteht aus einem durchsichtigen Gehäuse, auf dessen
Oberseite über einen magnetischen Deckel die Bohrnadel in jeder
Bohrposition ausgetauscht werden kann.

Der Resistograph R6 und das gesamte Zubehör (Bohrnadeln, Ersatzakku, Drucker,
etc. ) wird schaumstoffgelagert in einem stoßsicheren Plastikkoffer
transportiert. \footcite{rinn:anleitung_resi}

\subsubsection{Benutzung und Bedienung des Resistograph\SymbReg}
Bei der Benutzung des Resistograph\SymbReg\ Modell R6, geht es zum einen um
die eigentiche Bedienung am Holzkörper. Zum Anderen ist die korrekte 
Auswertung und Interpretation der Messergebnisse bzw. der Bohrprofile um
Aussagen zum Zustand des Holzes entlang des Bohrweges zu machen, unerlässlich.

Zur Durchführung einer Bohrung wird sowohl der Resistogpraph\SymbReg\ als auch
der Drucker eingeschaltet und via Bluetooth miteinander verbunden. Auf den
Displays der Unterseite des Geräts lassen sich vorab Projekt-, Objekt- und
Profilnummer einstellen. Jedes Profil erhält im geräteinternen Speicher einen
Dateinamen, der sich aus der Aneinanderkettung der Projekt-, Objekt- und
Profilnummer in der genannten Reihenfolge zusammensetzt. Um eine Bohrung zu
beginnen hält man das Bohrgerät unmittelbar an den Holzkörper und betätigt
den Kipphebel, der nach dem Haltegriff, in dem der Motor für den Vorschub
untergrabcht ist, als erstes angeordnet ist, um eine Bohrung zu beginnen. Das
Gerät bohrt automatisch die volle Länge von 50cm ins Holz. Es empfiehlt sich
bei Austritt aus dem Holzkörper entweder an dessen Ende oder in einem größeren
Hohlraum, die Bohrnadel sofort zurückzuziehen um ein Verbiegen der Bohrnadel
durch die Rotation der Nadel im Freien zu vermeiden.  Die
Eindringgeschwindigkeit der Nadel ins Holz wird automatisch an die Holzhärte
angepasst.

Während der Bohrung zeichnet der Drucker in Echtzeit ein mit der Bohrtiefe
übereinstimmendes Bohrprofil auf. Ein Ergebnis ist also unmittelbar nach der
Bohrung vorhanden und es können bereits am Baum oder an einer untersuchten
Holzkonstruktion erste Aussagen zum Holzzustand im untersuchten Bereich
gemacht werden.

\subsection{Jahrringanalyse}
Die Jahrringanalyse, die zum Vergleich mit den aus der Bohrwiderstandsmessung
gezogenen Jahrringdaten herangezogen wurde, wurde mit der
Lignostation\textsuperscript{TM} (im Folgenden LGS genannt) durchgeführt.

\subsubsection{Jahrringanalyse mit der Lignostation\textsuperscript{TM}}


\subsection{Versuchsaufbau}
Um einen möglichst realitätsnahen Versuch durchzuführen, wurden Proben von
Bestandesbäumen verwendet, die im Zuge eine Fällmaßnahme am 18.11.2017 gefällt
wurden. Die Bäume befanden sich an einem westexponierten Hang in 34225
Kirchbauna/Baunatal. Der genaue Standort kann Abbildung XXX entnommen werden.
Es wurden Proben von den drei Holzarten Nadelholz (NH), zerstreutporiges (ZH)
und ringporiges Laubholz (RH) entnommen.  Von diesen Holzarten wurden jeweils
drei Proben einer Kiefer, \textit{Pinus sylvestris} (NH), einer Hainbuche,
\textit{Carpinus betulus} (ZH) und einer Stieleiche, \textit{Quercus robur}
(RH) verwendet.

Bei den Proben handelt es sich um Baumscheiben, die alle vom Stammfuß
beginnend mit einer Motorsäge abgetrennt wurden wurden. Die
Versuchsdurchführung fand in zwei verschiedenen Zeiträumen in den
Räumlichkeiten der Firma Rinntech unter Aufsicht von Frank Rinn in Heidelberg
statt. 

Die Proben wurden beschriftet, vermessen und fotografisch dokumentiert.
Volumen, Gewicht und Größe können Tabelle XXX entnommen werden. Im nächsten
Schritt folgten an jeder Scheibe 2 Bohrungungen mittels Resistograph\SymbReg
R6 (im Folgenden R6 genannt). Der R6 wurde dabei so positioniert, dass die
Bohrnadel orthogonal auf die Jahrringegrenzen getroffen und durch den Kern,
einmal durch die gesamte Scheibe gebohrt ist. Um eine störungsfreie Kurve zu
erhalten, wurde sowohl die Holzscheibe als auch der R6 auf dem Versuchstisch
mittels Schraubzwingen fixiert. An jeder Scheibe wurden zwei Bohrungen, 90
Grad versetzt durchgeführt, um, wie bei der optischen Jahrringanalyse auch,
mindestens zwei Pfade von Jahrringen bis zum Kern verfolgen zu können. 

Damit die Scheiben in einem möglichst frischen Zustand bleiben, wurden die
Scheiben bis zur Bohrung in Frischhaltefolie eingewickelt. Sämtliche
Bohrungen wurden fotografisch dokumentiert. Jeder Ausdruck einer Bohrung wurde
über die durchbohrte Scheibe entlang des Bohrkanals gelegt und fotografiert.
Die Ein und Austrittsstellen der Bohrungen sind mit Schrauben und einer Nummer
am Eingang versehen. Die Nummer entspricht dabei der Bohrprofilnummer, die den
Dateinamen der abgespeicherten Bohrprofile im geräteinternen Speicher
entspricht.

Nach der Bohrung sind die Oberflächen der Scheiben parallel gefräst worden um
sie für die den Einsatz in der Lignosation verwenbar zu machen. Nach dem
Fräsen wurden die Scheiben an einem kühlen und trockenen Ort gelagert, bis die
Oberflächen der Scheiben getrocknet sind. Um eine optische Nachkontrolle unter
einem Mikroskop eines Jahrringmesstisches zu erlauben sind die Oberseiten der
Scheiben mit einem Schwingschleifer bis zur Körnung von P600 geschliffen
worden.

Lignostation

Kurven übereinander legen, vergleichen, zählen?

\section{Ergebnisse}
\section{Zusammenfassung}
\section{Anhang}

% Bibliographie
\addcontentsline{toc}{section}{Literaturverzeichnis}
\bibliographystyle{jurabib}
\bibliography{/Users/dda/Documents/Holzuntersuchung/biblio,/Users/dda/Documents/Jahrringanalyse/biblio,/Users/dda/Documents/Baumpflege/biblio,/Users/dda/Documents/Texte/biblio}




\end{document}
